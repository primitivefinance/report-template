\section{Primitive}\label{primitive}

Primitive is a team of deeply technical passionate indciduals, building
the future of finance.
You can find more information about us at \href{https://primitive.finance}{primitive.finance}.

\section{Introduction}\label{introduction}

This report is a security review of the Optimism Decentralization infrastructure: The dispute game.

\subsection{Dispute Game}\label{dispute-game}

The dispute game is a game theoretic game that will decentralize the technical infrastructure of the Optimism Stack.
This engadgement is designed to investigate these assumptions at both a game thoeritic level and a technical level.
The focus of the security review was trying to break the two assumptions

\begin{enumerate}
  \tightlist
  \item An honest player should never loose the dispute game.
  \item It should always be profitable to be an honest player.
\end{enumerate}

\emph{Disclaimer:} This security review does not guarantee against a
hack. It is a snapshot in time of brink according to the specific commit
by a three person team. Any modifications to the code will require a new
security review.

\subsection{Agent Based Modeling}\label{agent-based-modeling}

Arbiter uses agent based modeling with the rust evm to provide security
and risk analysis insights that are traditionally more difficult to
audit. Our agent architecture for the dispute game is as follows:

\begin{itemize}
\tightlist
\item
  Oracle Agent: Responsible for syncing the dispute game state by
  loading the latest claim and then solving the correct move for that
  claim by making an api call to durin. The oracle agent will then send
  honest moves to the honest agent.
\item
  Honest Agent: The Honest Agent is responsible for receiving the honest
  moves from the oracle agent and then acting on them in the dispute
  game.
\item
  Dishonest Agent: The Dishonest Agent is responsible for acting
  attempting to resolve an incorrect move in the dispute game, We will
  perturb the dishonest agent to look for insecurities in the protocol.
\end{itemize}

\subsection{Simulation Components}\label{simulation-components}

The system is composed of several agents and contracts. Below is a summary of the components:

\begin{itemize}
\item \textbf{Agents:}
  \begin{itemize}
  \item Oracle Agent: Responsible for syncing the dispute game state by loading the latest claim and then solving the correct move for that claim by making an api call to durin. The oracle agent will then send honest moves to the honest agent.
  \item Honest Agent: Responsible for receiving the honest moves from the oracle agent and then acting on them in the dispute game.
  \item Dishonest Agent: Responsible for attempting to resolve an incorrect move in the dispute game. We will perturb the dishonest agent to look for insecurities in the protocol.
  \end{itemize}
\item \textbf{Contracts:}
  \begin{itemize}
  \item Dispute Game: Holds all moves currently in the dispute game.
  \item Dispute Game Factory: Has pointers to all created dispute games.
  \end{itemize}
\item \textbf{Oracle:}
  \begin{itemize}
  \item Durin: A single oracle used in the system.
  \end{itemize}
\end{itemize}

% maybe put a nice photo from miro here

\subsection{Simulation Setup}\label{simulation-setup}
The state of the evm we simulated on was created by running a forge deploy script to an anvil instance and then dumping the state. The instance was then loaded into revm through arbiter.

\subsection{Risk Modeling}\label{risk-modeling}

We will perturb over the infinite space of dishost actors for the
dispute game. We will also perturb various L1 preposals. This willenable us to model the risk of the protocol in a more robust way. We will also measure the bond mechanics of the dispute game to ensure that there are no game theoretic attacks that can be made on the protocol.

\section{Findings}\label{findings}

\subsection{Critical Risk}\label{critical-risk}

\subsection{High Risk}\label{high-risk}

\subsubsection{Issue title (Only first word should be capitalized;
titles should never end with
punctuation)}\label{issue-title-only-first-word-should-be-capitalized-titles-should-never-end-with-punctuation}

\textbf{Severity:} High

\textbf{Context:}
\href{https://github.com/actuallink}{\texttt{Contract.sol\#L160-L165}}

\textbf{Description:}

\begin{minted}[]{solidity}
contract Test {
    ...
    // Code blocks must be indented with 4 spaces.
}
\end{minted}

\textbf{Recommendation:}

\begin{minted}[]{diff}
+ use diff syntax to describe what should be changed
- ...
\end{minted}

\textbf{Project:} Fixed in \href{Https://github.com/actuallink}{PR \#1}.

\textbf{Spearbit:} Resolved.

\subsection{Medium Risk}\label{medium-risk}

\subsection{Low Risk}\label{low-risk}

\subsection{Gas Optimizations}\label{gas-optimizations}

\section{Additional Comments}\label{additional-comments}

\textbackslash clearpage

\section{Appendix}\label{appendix}
